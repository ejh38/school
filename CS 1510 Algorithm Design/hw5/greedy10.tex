\documentclass[11pt]{article}
\usepackage{amsmath}
\begin{document}

\noindent 
CS 1510: Algorithm Design\\
Homework 5 - Greedy Problem 10\\
Zach Sadler, John Hofrichter\\
zps6@pitt.edu | jmh162@pitt.edu\\
September 8, 2013\\ 
\\ 
\\

\noindent \huge 
Problem 10a\\
\normalsize
\noindent 
We will disprove that the SJF algorithm is correct. Consider the input
\begin{align*}
	J_1 & = (0, 12)\\
    	J_2 & = (4, 10)\\
\end{align*}
When given this input, the algorithm will run $J_1$ until $t = 4$, then compare the two jobs. Since $x_1 = 12 > 10 = x_2$, the algorithm will switch and run $J_2$ until $t = 14$, then finish $J_1$ at $t = 22$. Thus the total completion time is $14 + 22 = 36$.\\
However, this is not the minimum completion time. Instead, one could run $J_1$ from $t = 0$ until $t = 12$, then $J_2$ from $t = 12$ until $t = 22$. In this case, the total completion time would be $12 + 22 = 34$.\\
Since $34 < 36$, the algorithm does not produce optimal output.\\
\\
\\
\noindent \huge 
Problem 10b\\
\normalsize
\noindent
We will prove that the algorithm SRPT (which we will call $G$) is correct.\\
Suppose towards contradiction that there is an input $I = J_1, J_2, \cdots, J_n$ (with each $J_i = (r_i, x_i)$ specifying the release time and size of the job) on which $G$ produces unacceptable output $G(I)$. Then by definition $G(I)$ is output which does not give the minimum total completion time.\\
Now define $Opt(I)$ to be an optimal output that agrees with $G(I)$ for more terms than any other optimal output. That is, beginning at time $t = 0$ and until time $t = k$, both $G(I)$ and $Opt(I)$ run the exact same jobs at each time interval. However, from time $(k, k+1), G$ runs $J_k$ and $Opt$ runs $J_a$ where $J_k \ne J_a$.\\
We know by definition of $G$ that
\[
x_k - y_{k, k} \le x_a - y_{a, k} 
\]
that is, that $J_k$ has no more remaining time than $J_a$.\\
Now we will construct an output $Opt'(I)$ which agrees with $G(I)$ for one more term than $Opt(I)$ and is still optimal. To do this, we will consider all future instances where job $J_k$ is run in $Opt(I)$ and label these instances $J_{k_1}, J_{k_2}, \cdots , J_{k_p}$. Similarly, we will label all instances where job $J_a$ is run in $Opt(I)$ as $J_{a_1}, J_{a_2}, \cdots , J_{a_q}$. We know that $p \le q$ since we have shown $J_k$ has no more remaining time than $J_a$, and so there are less unit intervals where $J_a$ is run than $J_k$.\\
Now we will begin swapping interval for interval segments of $J_k$ and $J_a$. That is, we will replace $J_{a_1}$ with $J_{k_1}$,  $J_{a_2}$ with $J_{k_2}$, and so on. Since $p \le q$ we know that we may run out of instances of $J_k$ to replace segments of $J_a$ with. If this happens, we will simply begin copying in the remaining instances of $J_a$.\\
Note that we know that both $J_k, J_a$ have been released since both algorithms have been identical to this point (the jobs have not already been completed), and the algorithms were able to select from already-released jobs and chose those two.\\
In the case where $p = q$, that is, $J_k$ and $J_a$ both had equal remaining time, we have simply swapped their positions in the schedule. Since they both had equal remaining time, doing this swap will not change the total completion time (it will increase the completion time of $J_a$ by some amount of units (say $m$) and decrease the completion time of job $J_k$ by the same number of units, thus resulting in $\delta = (+m) - m = 0$ no net change).\\
In the case where $p < q$, we will swap the locations of the first $p$ terms of $J_a$ and $J_k$. This will complete $J_k$'s runtime, granting it a smaller completion time in $Opt'$ than $J_a$ had in $Opt$. Then we will copy the remaining $q - p$ terms of $J_a$ into the remaining space left over since $J_k$ does not occupy the same space as $J_a$. Thus $J_a$ will finish in $Opt'$ at the same time that $J_k$ finished in $Opt$.\\
Because $J_k$ completes at an earlier time in $Opt'$ than $J_a$ did in $Opt$, and $J_a$ completes at the same time in $Opt'$ that $J_k$ did in $Opt$, and no other jobs are affected since we simply swapped the two in-place, we know that the total completion time of $Opt'$ is less than the total completion time of $Opt$.\\
Thus if either $p = q$ or $p < q$, we know that $Opt'$ has a total completion time which is no worse than $Opt$. so if $Opt'$ is optimal if $Opt$ is optimal. We also know that $Opt'$ agrees with $G$ for at least one more term than $Opt$, since $G$ ran $J_k$ at time $k$ and $Opt$ ran $J_a$. Thus we have a contradiction that $Opt$ is an optimal output which agrees with $G$ for more terms than any other optimal output and we are done.



\end{document}